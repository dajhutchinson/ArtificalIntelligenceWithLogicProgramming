\documentclass[11pt,a4paper]{article}

\usepackage[margin=1in, paperwidth=8.3in, paperheight=11.7in]{geometry}
\usepackage{amsfonts}
\usepackage{amsmath}
\usepackage{amssymb}
\usepackage{dsfont}
\usepackage{enumerate}
\usepackage{enumitem}
\usepackage{fancyhdr}
\usepackage{graphicx}
\usepackage{tikz}
\usepackage{changepage} 

\usepackage[utf8]{inputenc}
\usepackage{listings}
\usepackage{xcolor}

\lstset{
    frameround=fttt,
    language=Prolog,
    numbers=left,
    breaklines=true,
    keywordstyle=\color{blue}\bfseries, 
    basicstyle=\ttfamily\color{red},
    numberstyle=\color{black}
    }

\begin{document}

\pagestyle{fancy}
\setlength\parindent{0pt}
\allowdisplaybreaks

\renewcommand{\headrulewidth}{0pt}

% Cover page title
\title{Artificial Intelligence with Logic Programming - Notes}
\author{Dom Hutchinson}
\date{\today}
\maketitle

% Header
\fancyhead[L]{Dom Hutchinson}
\fancyhead[C]{Artificial Intelligence with Logic Programming - Notes}
\fancyhead[R]{\today}

% Counters
\newcounter{definition}[subsection]
\newcounter{example}[subsection]
\newcounter{notation}[subsection]
\newcounter{proposition}[subsection]
\newcounter{proof}[subsection]
\newcounter{remark}[subsection]
\newcounter{theorem}[subsection]

% commands
\newcommand{\dotprod}[0]{\boldsymbol{\cdot}}
\newcommand{\cosech}[0]{\mathrm{cosech}\ }
\newcommand{\cosec}[0]{\mathrm{cosec}\ }
\newcommand{\sech}[0]{\mathrm{sech}\ }
\newcommand{\prob}[0]{\mathbb{P}}
\newcommand{\nats}[0]{\mathbb{N}}
\newcommand{\cov}[0]{\mathrm{Cov}}
\newcommand{\var}[0]{\mathrm{Var}}
\newcommand{\expect}[0]{\mathbb{E}}
\newcommand{\reals}[0]{\mathbb{R}}
\newcommand{\integers}[0]{\mathbb{Z}}
\newcommand{\indicator}[0]{\mathds{1}}
\newcommand{\nb}[0]{\textit{N.B.} }
\newcommand{\ie}[0]{\textit{i.e.} }
\newcommand{\eg}[0]{\textit{e.g.} }
\newcommand{\X}[0]{\textbf{X}}
\newcommand{\x}[0]{\textbf{x}}
\newcommand{\iid}[0]{\overset{\text{iid}}{\sim}}
\newcommand{\proved}[0]{$\hfill\square$\\}

\newcommand{\definition}[1]{\stepcounter{definition} \textbf{Definition \arabic{subsection}.\arabic{definition}\ - }\textit{#1}\\}
\newcommand{\definitionn}[1]{\stepcounter{definition} \textbf{Definition \arabic{subsection}.\arabic{definition}\ - }\textit{#1}}
\newcommand{\proof}[1]{\stepcounter{proof} \textbf{Proof \arabic{subsection}.\arabic{proof}\ - }\textit{#1}\\}
\newcommand{\prooff}[1]{\stepcounter{proof} \textbf{Proof \arabic{subsection}.\arabic{proof}\ - }\textit{#1}}
\newcommand{\example}[1]{\stepcounter{example} \textbf{Example \arabic{subsection}.\arabic{example}\ - }\textit{#1}\\}
\newcommand{\examplee}[1]{\stepcounter{example} \textbf{Example \arabic{subsection}.\arabic{example}\ - }\textit{#1}}
\newcommand{\notation}[1]{\stepcounter{notation} \textbf{Notation \arabic{subsection}.\arabic{notation}\ - }\textit{#1}\\}
\newcommand{\notationn}[1]{\stepcounter{notation} \textbf{Notation \arabic{subsection}.\arabic{notation}\ - }\textit{#1}}
\newcommand{\proposition}[1]{\stepcounter{proposition} \textbf{Proposition \arabic{subsection}.\arabic{proposition}\ - }\textit{#1}\\}
\newcommand{\propositionn}[1]{\stepcounter{proposition} \textbf{Proposition \arabic{subsection}.\arabic{proposition}\ - }\textit{#1}}
\newcommand{\remark}[1]{\stepcounter{remark} \textbf{Remark \arabic{subsection}.\arabic{remark}\ - }\textit{#1}\\}
\newcommand{\remarkk}[1]{\stepcounter{remark} \textbf{Remark \arabic{section}.\arabic{remark}\ - }\textit{#1}}
\newcommand{\theorem}[1]{\stepcounter{theorem} \textbf{Theorem \arabic{section}.\arabic{theorem}\ - }\textit{#1}\\}
\newcommand{\theoremm}[1]{\stepcounter{theorem} \textbf{Theorem \arabic{section}.\arabic{theorem}\ - }\textit{#1}}

\tableofcontents

% Start of content
\newpage

\section{Introduction}

\definitionn{Types of AI}
\begin{enumerate}
	\item \textit{Weak AI} - Can solve a specific task.
	\item \textit{Strong AI} - Can solve general problems.
	\item \textit{Ultra Strong AI} - Can solve general problems \& explain why/what it is doing.
\end{enumerate}

\section{Logic Programming}

\definition{Logic Programming}
\textit{Logic Programming} is a \textit{declarative paradigm} where programs are concieved as a logical theory, rather than a set-by-step description of an algorithm. A Procedure call is viewed as a theorem which the truth needs to be established about. (\ie executing a programming is analogous to searching for truth in a system).\\

\remark{Variables}
In \textit{Logic Programming} a \textit{Variable} is a variable in the mathematical sense, that is they are are placeholders that can take on any value.\\

\remark{Machine Model}
A \textit{Machine Model} is an abstraction of the computer on which programs are executed. In \textit{Imperative Programming} we assume a dynamic, state-based machine model where the state of the computer is given by the contents of its memory \& a program statement is a transition from one statement to another. In \textit{Logic Programming} we do not assume such a dynamic model.

\subsection{Clausal Logic}

\notation{Variables \& Values}
\textit{Variables} are denoted by having a capitalised first letter, whereas \textit{values} are completly lowercase.\\

\definition{Clausal Logic}
\textit{Clausal Logic} is a formalism for representing \& reasoning with knowledge.
\begin{center}
\begin{tabular}{l|l}
\textbf{Keyword}&\textbf{Description}\\
\hline
{\lstinline!S:-C.!}&\textbf{If} condition {\lstinline!C!} holds \textbf{then} statement {\lstinline!S!} is true.\\
{\lstinline!S:-\+C.!}&\textbf{If} condition {\lstinline!C!} does \textbf{not} hold \textbf{then} statement {\lstinline!S!} is true.\\
{\lstinline!S:-C1,C2.!}&\textbf{If} conditions {\lstinline!C1!} \textbf{and} {\lstinline!C2!} both hold \textbf{then} statement {\lstinline!S!} is true.\\
{\lstinline!S:-C1;C2.!}&\textbf{If} at least one of {\lstinline!C1!} \textbf{or} {\lstinline!C2!} hold \textbf{then} statement {\lstinline!S!} is true.\\
{\lstinline|S:-C,!.|}&\textbf{Cut} the program after finding \textbf{first} \lstinline!S! where \lstinline!C! holds.
\end{tabular}
\end{center}

\definition{Facts \& Rules}
\textit{Facts} are logical formulae which are defined for explicit values \textbf{only}. \textit{Facts} denoted unconiditional truth.
\begin{center}{\lstinline!nearby(bond_street,oxford_circus).!}\end{center}
\textit{Rules} are logical formula which are defined in termms of variables (and explicit values). \textit{Rules} denote conditional truth.
\begin{center}{\lstinline!nearby(X,Y):-connected(X,Z,L),connected(Z,Y,L)!}\end{center}

\definition{Query, \lstinline!?-!}
A \textit{Query} asks a question about the knowledgebase we have defined. If we \underline{just} pass \textit{values} to a \textit{Query} then it shall simply return whether the statement is true or not. If we pass \textit{unbound variables} as well then it shall return values for the variable which make the statement true, if any exist.\\

\examplee{Query}
\begin{lstlisting}
?-nearby(bond_street,oxford_circus)
?-nearby(bond_street,X)
\end{lstlisting}
(1) will return \textit{true} if we have defined \lstinline!bond_street! to be near to \lstinline!oxford_circus!.\\
(2) will return all the values of \lstinline!X! (\ie stations) which are near to \lstinline!bond_street!.\\

\definition{Resolution}
In order to answer a query \lstinline!?-Q1,Q2,...! find a rule \lstinline!A:-B1,...,Bn! such that \lstinline!A! matches with \lstinline!Q1! then proceed to answer \lstinline!?-B1,...,Bn,Q2,...!.\\
This is a \textit{procedural interpretation} of logical formulae \& is what allows \textit{Logic} to be a programming language.\\

\definition{Functor}
\textit{Functor}s provide a way to name a complex object composed of simpler objects \& are never evaluated to determine a value.
\begin{lstlisting}
reachable(X,Y,noroute):-connected(X,Y,L)
reachable(X,Y,route(Z,R)):-connected(X,Z,L),connected(Z,Y,R)
\end{lstlisting}
Querying \lstinline!?-reachable(oxford_circus,tottenham_court_road,R)! will return a route \lstinline!R! which connects the two stations, on a single line.\\
The above definition can be read as \lstinline!X! is reachable from \lstinline!Y! if they are connected \textbf{or} if there exists a station \lstinline!Z! which is connected.\\

\definition{List Functor, \lstinline!.!}
The \textit{List Functor} takes two arguments, one on each side, and has terminator \lstinline![]!.
\begin{center}
[a,b,c]$\equiv$\lstinline!.(a,.(b,.(c,[])))!
\end{center}
Alternatively we can use a pipe to distinguish between a value and the rest of the list
\begin{center}
\lstinline![X,Y|R]!
\end{center}

\remark{Logical Formulae}
\textit{Logical Formulae} determine what is being modelled \& the set of formulae that can be derived by applying inference rules.\\
\ie \textit{Logical Formulae} have both declarative (what is true) \& procedural (how the truth is reached) meaning.\\
Changing the order of statements in a clause does not change the declarative meaning ($3=2+1\equiv3=1+2$) but does change the procedular meaning \& thus chanes what loops we may get stuck in, what solutions are found first \& how long execution takes.\\
\nb The procedular meaning of \textit{Logical Formulae} is what allows them to be used as a programming language.

% CHAPTER 3.1
\subsection{SLD-Resolution}

\definition{SLD-Resolution}
\textit{SLD-Resolution} is the process $\mathtt{Prolog}$ uses to resolve a query.
\begin{enumerate}
	\item \textit{\textbf{S}election Rule} - Left-to-right (Always take the left branch if it is there).\\
	This is equivalent to reading clauses top to bottom in a $\mathtt{Prolog}$ program.
	\item \textit{\textbf{L}inear Resolution} - The shape of the proof trees obtained.
	\textit{\textbf{D}efinite Clauses}
\end{enumerate}
\nb These rules vary for different languages.\\

\definition{SLD-Tree}
\textit{SLD-Trees} are graphical representations of \textit{SLD-Resolution}.\\
They are not the same as a proof tree as only the resolvents are shown (no input clauses or unifiers) andit contains every possible resolution step, leading to every leaf on an \textit{SLD-Tree} being an empty clause $\square$.\\
The left-most leaf in an \textit{SLD-Tree} will be returned as the first solution, if we end up in an infinite sub-tree then no solution will ever be returned.\\
Non-leaf nodes in \textit{SLD-Trees} are called \textit{choice points}.
\nb If a leaf is a failure then we underline it.\\

\proposition{Traversing an SLD-Tree}
$\mathtt{Prolog}$ traverses \textit{SLD-Trees} in a \textit{depth-first} fashion, backtracking whenever it reaches a success or a failure. This can be visualised as following the left branch of the tree whenever it is avaiable. This leads to $\mathtt{Prolog}$ being incomplete (does not always return all solutions) as once it is in an infinite sub-tree it has no way of escaping.\\

\remark{Backtracking}
Since we need to be able to \textit{Backtrack} after finding successes or failures, we need to store all previous resolvents which have not been fully explored yet as well as a pointer to the most recent program clause that has been tried.\\
Due to the depth-first nature of $\mathtt{Prolog}$ this can be stored as a \textit{stack}.\\

\remark{Order of clauses}
Due to the \textit{Selection Rule} being left-to-right you should \textit{put non-recursive clauses before recursive clauses} to ensure that solutions are found before infinite sub-trees.\\

\definition{Transitivity}
\textit{Transitivity} is a property of predicate that states
$$X=Y\ \&\ Y=Z\implies X=Z$$
\textit{Transitivity} clauses can cause infinite loops is not implemented carefully.\\

\remark{Why not use Breadth-First Search?}
\textit{Breadth-First Search} requires for each branch at a given level to be tracked (rather than just one in \textit{Depth-First Search}), this requires more memory and was deemed memory-inefficient by $\mathtt{Prolog}$ developers.\\

\remark{Queries with no solution}
Some queries do not have a solution as the \textit{SLD-Tree} they produce is infinite, no matter how it is read \& thus an answer can never be resolved.\\
Consider
\begin{lstlisting}
brother_of(paul,peter).
brother_of(X,Y):-brother_of(Y,X).
\end{lstlisting}
The query \lstinline!brother_of(peter,maria)! has no answer since its \textit{SLD-Tree} is infinite.\\

% CHAPTER 3.2
\definition{Cut, \lstinline|!|}
A \textit{Cut} says that once it has been reached, stick to all the variable substitutions found after entering its clause. (\ie Do not backtrack past it?).\\
\textit{Cuts} have the effect of pruning the \textit{SLD-Tree}.\\
\textit{Cuts} which \underline{don't} remove any successful branches are called \textit{Green Cuts} and are harmless.\ \textit{Cuts} which \underline{do} remove successful branches are caleld \textit{Red Cuts} and affects the procedular meaning of the process.\\
\nb \textit{Cuts} make the left sub-tree deterministic.\\

\notationn{In SDL-Trees we put a box around subtrees which are pruned by \lstinline|!|.}

% Chapter 3.3
\subsection{Negation as Failure}

\definition{Netgation as Faliure}
\textit{Negation as Failure} is an interpretation of the semantics for failure.\\
It states if we \underline{cannot} prove \lstinline!q! to be true then we take \lstinline!not(q)! to be true.\\
\ie If we don't have enough information to state where \lstinline!q! is true then we assume it to be false.\\
\eg Negation as failure would resolve that \textit{``I cannot prove God exists, therefore God does not exist.''}.\\

\proposition{Altering Negation as Failure}

\proposition{Cut as Failure}
Suppose we have a program of the following form
\begin{lstlisting}
p:-q,!,r.
p:-s.
\end{lstlisting}
We can interpret it as
\begin{lstlisting}
p:-q,r.
p:-not_q,s.
\end{lstlisting}
where \lstinline!not_q:=q,fail.!.\\

\definition{Negation Function, \lstinline!not()!}
Let \lstinline!Goal! be a literal we wish to negate
\begin{lstlisting}
not(Goal):-Goal,!,fail.
not(Goal).
\end{lstlisting}

\definition{Call Function, \lstinline!call()!}
TODO\\

\remark{\lstinline!not()! \& \lstinline!call()! are called meta-predicates as they take formulas as arguments.}

\remark{Negation as Failure is \textbf{not} Logical Negation}
If we cannot prove predicate \lstinline!q! we know that \lstinline!q! \underline{is not} a logical consequence of the program, but that does not mean that its negation \lstinline!:-q! \underline{is} a logical consequence of the program.\\
This is acceptable reasoning in some scenarios, but not others.\\

\definition{Logical Negation}
\textit{Logical Negation} can only be expressed by indefinite clauses, as below
\begin{lstlisting}
p;q,r.
p;q:-s.
s.
\end{lstlisting}

%CHAPTER 3.4

\definition{Conditionals, \lstinline!if_then_else()!}
We give the following definition where if \lstinline!S! holds then \lstinline!T! is executed, else \lstinline!U! is executed.
\begin{lstlisting}
if_then_else(S,T,U):-S,!,T/
if_then_else(S,T,U):-U.
\end{lstlisting}

\notation{Implication, \lstinline!->!}
We can nest \lstinline!if_then_else()! to add more \textit{elif} clauses but is clumsy notation.\\
Instead we simplify \lstinline!if_then_else(S,T,U).! to \lstinline!S->T;U.!.\\
This can be nested as \lstinline!P->Q;(R->S;T)!. The keyword \lstinline!otherwise! can be used instead of \lstinline!true! as it can be more readable.

% CHAPTER 3.5

\end{document}